\documentclass[11pt,a4paper]{article}

\usepackage[T1]{fontenc}
\usepackage[utf8]{inputenc}
\usepackage[frenchb]{babel}

\usepackage{fancyhdr} % headers
\usepackage[usenames,dvipsnames]{color} % colors
\usepackage{graphicx} % images
\usepackage{listings} % source code
\usepackage{titling} % meta-infos
\usepackage{courier} % courier font
\usepackage{fullpage} % full page layout
\usepackage{titlesec} % title customization
\usepackage{parskip} % paragraphs spacing
\usepackage{amsmath}
\usepackage{tikz}
\usepackage{siunitx}
\usepackage{tabularx}
%\usepackage{showframe} % layout debug

\usepackage{float}
\restylefloat{figure}
\usepackage[nottoc,numbib]{tocbibind}

\topmargin -10mm
\headsep 8mm
\headheight 10mm

\linespread{1.0}
\renewcommand{\arraystretch}{1.3}

\setlength{\unitlength}{1cm}
\setlength{\droptitle}{5cm}

\pagestyle{fancy}
\fancyhf{}
\cfoot{\thepage}

\def \doccourse { INF2-B -- Présentation }
\def \doctitle {Gestion de versions avec Git}
\author{Bastien Clément \and Christophe Peretti}

\rhead{\theauthor \\ \today}
\lhead{\doccourse \\ \doctitle }
\title{{\normalsize \doccourse} \\ \doctitle }

\begin{document}
\pagenumbering{gobble}

\maketitle

\vspace{3cm}
\begin{center}
\includegraphics[width=5cm]{git_logo}
\end{center}

\pagebreak
\pagenumbering{roman}

\tableofcontents

\pagebreak
\pagenumbering{arabic}

\section{Introduction à la gestion de versions}

Git est un logiciel de gestion de versions libre et open-source.
Avant de s'y attaquer directement, il est nécessaire de commencer par une petite introduction sur ce concept de gestion de versions.
De quoi s'agit-il et pourquoi est-ce intéressant dans le domaine de la programmation logiciel ?

Basiquement, un logiciel de gestion de versions se charge d'enregistrer les modifications apportées à un ensemble de fichiers au cours du temps.
Par la suite, il permet de naviguer dans cet historique pour consulter ou récupérer d'anciennes versions d'un fichier particulier.
Dans notre cas, nous l'utiliserons typiquement pour conserver l'historique du code source de nos programmes.

\subsection{Gestion de versions locale}

Une façon de mettre en place un système de gestion de versions très basique serait d'effectuer périodiquement des copies du projet en les annotant avec la date et l'heure.
Il serait alors possible de retourner dans une copie particulière pour avoir accès à une version antérieur d'un fichier.

Des logiciels automatiques sur le même principe sont même intégrés directement aux système d'exploitation comme par exemple \textit{Time Machine} sur OS X et \textit{File History} sur Windows.
Ces sauvegardes sont cependant effectuées automatiquement à intervalles fixes et ne correspondent pas aux étapes du développement du logiciel.

Les logiciels spécialisés (tel que \textit{rcs}, développé depuis 1982) permettent à l'inverse d'effectuer des sauvegardes à des moments particuliers, typiquement lorsqu'une fonctionnalité du programme est terminée et prête à être enregistrée.
Ces logiciels fournissent ainsi un niveau de \textit{sécurité} et d'\textit{historique}, constituant les premières facettes de la gestion de versions.

\subsection{Gestion de versions centralisée}

Le développement de logiciels, pour tout projet un minimum sérieux, implique généralement la collaboration de plusieurs développeurs.
Même dans le cadre de nos laboratoires encore relativement simples, nous avons pu constater les difficultés liées au développement à plusieurs.
Comment rester à jour sur le travail effectué par notre binôme ?
Comment garder une copie cohérente du projet lorsqu'un fichier est modifié par plus d'une personne au même moment ?

Les logiciels de gestion de versions centralisés règlent ce problème en se basant sur un \textit{dépôt central} dans lequel les fichiers constituant la version de référence du logiciel sont enregistrés.
C'est le cas par exemple du logiciel \textit{Subversion} (souvent abrégé \textit{svn}).

Lorsqu'un développeur souhaite modifier le logiciel, il commence par en effectuer une copie sur sa machine (opération \textit{checkout}), modifie et enregistre son travail localement (opération \textit{commit}), puis il termine en envoyant ses modifications sur le dépôt central (opération \textit{update}) pour qu'elles soient intégrées définitivement au logiciel.
Il récupère au passage les modifications des autres développeurs et est informés des situations de conflits, c'est à dire deux modifications différentes sur la même partie de code, qu'il doit résoudre avant de pouvoir continuer.

Dans cette configuration, le dépôt central constitue un \textit{single point of failure}.
S'il est inaccessible, la collaboration et l'envoi des modifications deviennent temporairement impossible.
Si un problème plus grave survient, en l'absence de sauvegarde récente, une partie ou l'ensemble de l'historique du projet peut être définitivement perdu ou corrompu.

\subsection{Gestion de versions distribuée}

À l'inverse, dans un système décentralisé tel que Git (mais aussi \textit{Mercurial} ou \textit{Bazaar}), lorsqu'une copie locale du projet est effectuée, ce n'est pas uniquement les dernières versions des fichiers qui sont récupérées, mais l'ensemble de l'historique du dépôt.
On parle alors de \textit{clônage}.

Il n'y a ainsi plus de dépôt central à proprement parler et chaque développeur possède naturellement un copie complète de l'historique du projet.
La majorités des opérations peuvent ainsi s'exécuter hors-ligne avec d'excellentes performances.

L'absence de dépôt central permet également d'organiser le développement de façon très flexible selon ce qui est le plus adapté à chaque situation.
En utilisant un dépôt disponible en ligne, Git peut même reproduire l'organisation typique d'un système centralisé.

\begin{figure}[ht]
\begin{center}
\includegraphics[width=11.5cm]{img_dvcs}
\caption{Exemple d'organisation décentralisée avec un \textit{responsable d'intégration}}
\end{center}
\end{figure}

Dans l'exemple donné, les développeurs clônent le dépôt de référence et effectuent leurs modifications localement.
Ils les publient ensuite dans un dépôt publiquement accessible et demande au responsable de récupérer les changements pour les intégrer dans le dépôt officiel après validation.

\section{Git}

\subsection{Histoire}

L'histoire du projet Git est étroitement liée au développement du noyau Linux, développés tous deux par Linus Torvalds.

Dans un premier temps, de 1991 à 2002, les développeurs et contributeurs de Linux n'utilisaient pas de logiciels particuliers et échangeaient leur travail sous forme de patches et d'archives.
Face aux difficultés engendrées par la croissance du projet, la décision d'utiliser le logiciel de gestion de versions propriétaire BitKeeper est prise.
L'éditeur proposait alors une licence gratuite permettant, sous certaines conditions, d'utiliser gratuitement une version limitée de son logiciel pour le développement de logiciels libre.

En 2005, un conflit entre un développeur indépendant de l'organisation OSDL (\textit{Open Source Development Labs}) et la société éditrice BitMover conduit à la résiliation de la licence particulière pour les logiciels libres.
Bien que BitMover proposent alors de continuer à fournir gratuitement son logiciel à certains développeurs de Linux, elle le refuse à Linus Torvalds, également membre de l'OSDL.
L'utilisation de BitKeeper est alors abandonnée et le projet Git est lancé avec pour but de devenir le nouveau logiciel de gestion de versions utilisé par les développeurs Linux.

\subsection{Caractéristiques}

Destiné à l'un des plus grand projet open-source existant, Git a été conçu dès le départ avec beaucoup d'exigences sur la base de ce qui avait été appris avec l'utilisation de BitKeeper:

\begin{enumerate}
	\item {Entièrement distribué}, 
	il n'y a aucun dépôt fondamentalement plus important,
	
	\item {Gestion du développement non-linéaire},
	même avec des milliers de branches simultanées,
	
	\item {Capable de supporter de très gros projets}
	tel que Linux tout en conservant vitesse d'exécution et optimisation de l'espace disque.
\end{enumerate}

Au fil du temps, Git a évolué en conservant ces caractéristiques faisant de lui un excellent logiciel dans son domaine.
Il est maintenant largement utilisé en dehors du développement de Linux.

\subsection{Chaîne de révisions}

Le fonctionnement de Git peut être comparé à celui d'un système de fichiers (tel que celui qui gère les fichiers d'un disque dur) sur lequel on a greffé des fonctionnalités de gestion de versions.
En effet, un dépôt n'est rien d'autre qu'une structure de répertoires et de fichiers associés à leur contenu, accompagnée d'un historique de modifications.

Lorsque l'utilisateur enregistre une nouvelle version de son projet (appelée \textit{commit} ou \textit{révision}), un \textit{snapshot} de l'état actuel des fichiers est enregistré dans l'historique du dépôt.
Par optimisation, lorsqu'un fichier n'a pas été modifié, le contenu du snapshot précédent est réutilisé tel quel. Un lien est également créé indiquant le commit antérieur, formant ainsi l'historique.

La dernière version des fichier sur lesquels travail l'utilisateur (le \textit{répertoire de travail}) sont à la fin de la chaine et formeront la prochaine révision du projet lors du prochain commit.

\begin{figure}[ht]
\begin{center}
\includegraphics[width=11cm]{img_snapshots}
\caption{L'historique du dépôt est une séquence de snapshots des fichiers du projet}
\end{center}
\end{figure}

\subsection{Stockage adressé par contenu}

La base de données de Git est un \textit{stockage adressé par contenu} (en anglais \textit{Content Addressable Storage}). C'est à dire une base de données qui identifie les informations sur la base de leur contenu plutôt que d'une clé ou d'un nom.
Il est donc nécessaire de connaître le contenu de ce que l'on souhaite récupérer pour pouvoir l'obtenir ce qui peut sembler paradoxal.

Le système se base sur la fonction de hachage cryptographique SHA-1 qui associe à toute information une emprunte (ou \textit{hash}) de 40 caractères.
Tout ce qui est contenu dans l'historique est premièrement hashé puis enregistré sur le disque à un emplacement qui dépend de la valeur obtenue.
Pour retrouver cette information, il est indispensable d'enregistrer ce hash quelque part, constituant ainsi un \textit{pointeur} contenant l'\textit{adresse} de cette information.

Il y a principalement trois types d'objets dans la base de données de Git:

\begin{enumerate}
	\item Les {\bf commits},
	ils contiennent l'ensemble des méta-informations d'une révision particulière tel que son auteur, la date et une description mais aussi l'adresse d'un \textit{tree} représentant le dossier principal du projet et l'adresse du commit qui le précède.
	
	\item Les {\bf trees},
	ils listent l'ensemble des fichiers et sous-dossiers d'un répertoire ainsi que l'adresse de leur contenu.
	L'adresse d'un sous-dossier sera l'adresse d'un autre \textit{tree}.
	
	\item Les {\bf blobs},
	ils sont les fichiers eux-mêmes.
\end{enumerate}

\begin{figure}[H]
\begin{center}
\includegraphics[width=8cm]{img_structure} \cite{progit}
\caption{Structure interne d'un dépôt Git avec \textit{commits}, \textit{trees} et \textit{blobs}}
\end{center}
\end{figure}

Puisque l'adresse d'un blob dépend de son contenu, il est évident qu'une modification provoquera un changement de cette adresse et que la propagation de ce changement dans les \textit{trees} entrainera un nouvel arbre raçine à une adresse différente, correspondant à une nouvelle révision.

Cette structure entraine naturellement deux propriétés très appréciables:

\begin{enumerate}
	\item \textbf{Déduplication}, 
	si une information est identique à une autre déjà contenue dans la base de données, leurs adresses seront identiques et le contenu existant sera simplement réutilisé.
	
	\item \textbf{Intégrité},
	l'adresse d'une information étant directement liée à son contenu, il est impossible qu'une corruption du dépôt ne passe inaperçue.
	Chaque bit d'information récupéré de la base de données est toujours identique à celui qui y a été enregistré.
\end{enumerate}

\section{Branches}

Jusqu'à maintenant, nous n'avons considéré que l'enregistrement d'une séquence linéaire de modifications. La vraie force de Git réside dans son excellent support de \textit{branches} représentant des modifications parallèles et indépendantes des fichiers du projet.

\subsection{Références}

Dans la section précédente, nous avons vu comment l'historique du dépôt était organisé en une chaîne de commits. Il reste un dernier problème: comment obtenir l'adresse du dernier commit permettant ensuite d'accéder à l'ensemble de la structure ?

Cette information essentielle, appelée \textit{référence}, est enregistrée en dehors de la base de données principale et associe un nom à un commit spécifique. Chaque branche de développement correspondent donc simplement à une référence particulière indiquant le dernier commit de cette branche. Par défaut, un dépôt Git ne contient qu'une seule branche appelée \textit{master}. Il est également possible pour plusieurs références de pointer vers le même commit.

\begin{figure}[h]
\begin{center}
\includegraphics[width=6cm]{img_refs}
\caption{Deux références, \textit{master} et \textit{exemple}, pointant vers le même commit}
\end{center}
\end{figure}

Lors de l'enregistrement d'une nouvelle révision, la référence de la branche courante est avancée pour pointer sur le dernier commit de la chaîne.

\begin{figure}[h]
\begin{center}
\includegraphics[width=8cm]{img_refs2}
\caption{La référence de la branche \textit{exemple} est avancée suite à un commit}
\end{center}
\end{figure}

Il est à tout moment possible de changer la branche sur laquelle on travail. Les fichiers du répertoire de travail sont alors mis à jour pour refléter l'état dans lequel ils étaient lors du dernier commit de la branche en question.

\subsection{Divergences}

Que ce passe-t-il à présent si des modifications sont enregistrées sur la branche \textit{master} ? Nous sommes alors face à une situation de divergence où le développement du projet s'est divisé en deux branches distinctes et indépendantes.

\begin{figure}[h]
\begin{center}
\includegraphics[width=8cm]{img_divergence}
\caption{Les branches \textit{master} et \textit{exemple} ont divergé}
\end{center}
\end{figure}

Il est par exemple possible d'utiliser ce mécanisme pour développer deux fonctionnalités indépendantes sans que le développement de l'une d'elles n'impacte le développement de l'autre.

C'est aussi implicitement la situation qui survient lorsque deux développeurs travaillent en parallèle sur la même version du logiciel, dans ce cas, il s'agit d'une divergence entre une branche et elle-même, mais dans un dépôt différent.

\subsection{Méthodes de fusion}

Si les branches ont un intérêt, c'est parce qu'il est possible à un certain moment de les rassembler, qu'elles soient divergentes ou non, pour n'en former plus qu'une. Il est alors question de \textit{fusion} des branches. Pour ce faire, Git possède trois mécanismes principaux.

\subsubsection{\textit{Fast-forward}}

Un \textit{fast-forward} a lieu en l'absence de divergences. La référence de la branche \textit{en retard} est simplement avancé pour pointer vers le dernier commit.

\begin{figure}[h]
\begin{center}
\includegraphics[width=8cm]{img_fastf}
\caption{\textit{Fast-forward} de la branche \textit{master} lors d'une fusion sans divergences}
\end{center}
\end{figure}

\subsubsection{\textit{Three-way merge}}

Lorsqu'une divergence a eu lieu, Git utilise alors la technique du \textit{three-way merge} pour rassembler les deux branches automatiquement. 
Ce mécanisme requiert l'existence d'un \textit{ancêtre commun}, ce qui est en principe toujours le cas, sur lequel se baser pour déterminer le résultat de la fusion. 
Le résultat est un nouveau commit qui a la particularité de posséder deux parents et qui combine l'ensemble des modifications des deux branches.

\begin{figure}[ht]
\begin{center}
\includegraphics[width=13cm]{img_twmerge}
\caption{\textit{Three-way merge} de la branche \textit{exemple} dans la branche \textit{master}}
\end{center}
\end{figure}

Git peut également généraliser le principe à un nombre arbitraire de branches. La méthode est alors appelée \textit{octopus merge} et permet de rassembler d'un coup plusieurs branches divergentes.

\subsubsection{\textit{Rebasing}}

Cette dernière opération un peu particulière n'est pas strictement une fusion. Elle consiste à recopier les commits d'une branche à la suite d'une autre.

\begin{figure}[ht]
\begin{center}
\includegraphics[width=13cm]{img_rebase}
\caption{\textit{Rebasing} de la branche \textit{exemple} sur la branche \textit{master}}
\end{center}
\end{figure}

Il est important de noter que cette opération requiert la modification du commit original, plus particulièrement de la référence vers le commit antérieur.
Puisque les informations sont modifiées, son adresse est nécessairement différente. Il s'agit en tout points d'un commit différent.

Cette méthode est généralement utilisée pour intégrer les modifications d'un autre dépôt dans le notre, tout en évitant la création d'un commit de fusion.
Elle permet de conserver un historique fortement linéaire même lors de modifications apportées en parallèle.

\subsection{Conflits}

Il peut arriver que deux branches divergentes contiennent des modifications différentes apportées à la même partie d'un fichier. Dans une telle situation, l'algorithme de fusion automatique n'est pas capable de déterminer le résultat approprié.
Peut-être que l'une des deux modifications devrait être remplacée par l'autre ?
Peut-être que les deux devraient plutôt être combinées ?

Dans ce cas, Git annonce un \textit{conflit} et arrête temporairement l'opération.
Il est alors nécessaire d'intervenir manuellement pour résoudre l'ambiguïté avant de pouvoir terminer la fusion ou le \textit{rebasing}.
Synchroniser régulièrement sa branche locale avec la branche de référence est une méthode pour réduire les risques de collisions et simplifier leur résolution.

\section{Conclusion}

Cette présentation aura permis de découvrir une vue d'ensemble du logiciel de gestion de versions Git ainsi que des exemples d'organisation et d'utilisation illustrant sa grande flexibilité. 

La puissance de Git réside entre autre dans sa manière d'organiser les révisions d'un fichier, ainsi que celle de fusionner avec une incroyable efficacité les différentes branches de celui-ci. Nous avons vu que la seule chose qu'il reste à faire manuellement aux utilisateurs est la gestion des conflits, et cette tâche ne se produit, en principe, que très rarement lorsque les développeurs sont bien organisés.

On remarquera enfin que l'utilisation d'un tel outil devient indispensable lors d'un développement logiciel, que ce soit pour garder une trace des différentes modifications qui y ont été apportées, ou plus généralement dans le cas d'une collaboration de plusieurs développeurs.

Vous trouverez en complément de cette présentation quelques éléments d'information concernant le site internet GitHub, principale plateforme d'utilisation du logiciel Git.

\pagebreak
\pagenumbering{alph}
\section{Compléments}

\subsection{Téléchargement et installation de Git}

Git peut être téléchargé depuis le site officiel du projet {\tt <http://git-scm.com/>}.

Le logiciel est disponible pour tous les systèmes d'exploitation courants.

\subsection{GitHub}

Il est difficile aujourd'hui de présenter Git sans mentionner également le site internet GitHub.

Lancé le 10 avril 2008, c'est un service indépendant d'hébergement et de gestion de développement de logiciels basé sur Git qui compte aujourd'hui plus de 9 millions d'utilisateurs.

GitHub fait partie des facteurs ayant contribué à l'adoption rapide de Git dans le monde de l'open source et est aujourd'hui une plateforme incontournable, régulièrement utilisée par de grands noms tel que Google, Microsoft et Facebook. 

Destinée avant tout aux développeurs, l'interface est très simple et épurée, mettant l'accent sur le code source en tant qu'élément central du développement.

\subsection{Offre GitHub pour les étudiants}

L'hébergement de dépôt sur GitHub est entièrement gratuit pour les projets open-source.
En revanche, il est nécessaire souscrire à un abonnement pour pouvoir créer des dépôts privés dont l'accès est limité aux personnes autorisées.

Heureusement, les personnes possédant une adresse e-mail provenant d'une université reconnue (dont la HEIG-VD fait partie!) peuvent demander un plan \textit{Micro} gratuit pendant toute la durée des études, soit 5 dépôts privés avec un nombre illimités de collaborateurs par dépôt.

L'inscription se fait à l'adresse {\tt <https://education.github.com/>} et est accompagnée d'autres offres pour former le {\it Student Developer Pack}.

\subsection{Client GitHub}

GitHub propose également un client Git permettant d'effectuer les tâches les plus courantes de façon visuelle et intuitive, sans utiliser par la ligne de commandes.
Il est aussi directement lié à la plateforme en ligne permettant le clônage ou la publication de dépôts locaux en un clic.
Il contient une version intégrée de Git, il n'est donc pas nécessaire de l'installer au préalable.

Il est peut être téléchargé à l'adresse

\begin{itemize}
	\item {\tt <https://windows.github.com/>}, pour le client Windows
	\item {\tt <https://mac.github.com/>}, pour le client Mac OS
\end{itemize}

Bien qu'il remplisse très bien son rôle dans la plupart des cas, certaines erreurs ou situations très particulières peuvent nécessiter l'utilisation de commande dans le terminal pour rétablir le bon fonctionnement.
Quelques connaissances de Git restent donc très utiles.

\pagebreak
\begin{thebibliography}{1}

\bibitem{progit}
	CHACON, Scott et STRAUB, Ben. {\em Pro Git} (2e éd). Apress. 2014. \\
	Disponible en ligne: {\tt http://git-scm.com/book/en/v2}

\bibitem{about}
	{\em About Git}. Site du projet Git ({\tt http://git-scm.com/about}). \\
	Consulté le 3 juin 2015.

\bibitem{wiki-git}
	{\em Git}. Wikipedia ({\tt http://en.wikipedia.org/wiki/Git}). \\
	Consulté le 3 juin 2015.

\bibitem{wiki-bk}
	{\em BitKeeper}. Wikipedia ({\tt http://en.wikipedia.org/wiki/BitKeeper}). \\
	Consulté le 3 juin 2015.

\bibitem{wiki-gh}
	{\em GitHub}. Wikipedia ({\tt http://en.wikipedia.org/wiki/GitHub}). \\
	Consulté le 3 juin 2015.
	
\end{thebibliography}

\end{document}