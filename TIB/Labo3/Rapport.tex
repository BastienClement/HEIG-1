\documentclass[11pt,a4paper]{article}

\usepackage[T1]{fontenc}
\usepackage[utf8]{inputenc}
\usepackage[frenchb]{babel}

\usepackage{fancyhdr} % headers
\usepackage[usenames,dvipsnames]{color} % colors
\usepackage{graphicx} % images
\usepackage{listings} % source code
\usepackage{titling} % meta-infos
\usepackage{courier} % courier font
\usepackage{fullpage} % full page layout
\usepackage{titlesec} % title customization
\usepackage{parskip} % paragraphs spacing
\usepackage{amsmath}
\usepackage{tikz}
\usepackage{siunitx}
%\usepackage{showframe} % layout debug

\usepackage{float}
\restylefloat{figure}

\topmargin -10mm
\headsep 5mm
\headheight 10mm

\linespread{1.1}
\renewcommand{\arraystretch}{1.3}

\setlength\parindent{0pt}
\setlength{\unitlength}{1cm}
\setlength{\droptitle}{-1.6cm}

\pagestyle{fancy}
\fancyhf{}
\cfoot{\thepage}

\def \doccourse { TIB1-B }
\def \doctitle {Labo : Mécanismes LAN}
\author{Bastien Clément \and Christophe Peretti}

\renewcommand{\thesection}{Objectif \arabic{section} :}
\renewcommand{\thesubsection}{\arabic{section}.\arabic{subsection}}

\rhead{\theauthor \\ \today}
\lhead{\doccourse \\ \doctitle }
\title{{\normalsize \doccourse} \\ \doctitle }

\begin{document}

\maketitle
\vspace{1em}

\section{Adresses IP et MAC et le protocole ARP}

Objectif et critère de succès

\subsection{Protocole ARP}

Le protocole ARP permet la traduction des adresses IP locales en adresses MAC. En effet, pour pouvoir indiquer correctement la machine de destination du paquet, notre ordinateur a besoin de connaitre la machine à qui il doit s'adresser.

...

Le résultat d'une requête ARP est conservée en cache un certain temps pour être réutilisé tant que la communication est fonctionnelle.

\subsection{Pourquoi des adresses MAC sont-elles nécessaires ?}

\section{Hub et switch}

Objectifs et critères de succès

\subsection{Différentes méthodes d'acheminement}

Un hub est un équipement réseau de niveau 1 (couche physique). Il n'effectue aucun traitement sur l'information qu'il reçoit et ce contente de réémettre la même trame sur toutes ses interfaces. Un réseau créé à partir de hubs ne contient qu'un seul \textit{domaine de collision}. C'est à dire que dès qu'une des machines le composant communique, aucune autre ne peut utiliser le réseau.

À l'inverse, un switch est un équipement de niveau 2 (couche liaison) qui analyse les trames reçues pour en extraire l'adresse MAC de destination et ne transmettre la trame que sur l'interface correspondant à ce destinataire précis.

\subsection{Captures effectuées}

Pour observer cette différence, nous avons réitéré la capture de l'objectif 1, une première fois avec un hub, puis avec un switch. Dans les deux cas la capture est effectuée depuis un second ordinateur.

\subsection{Auto-apprentissage du switch}

Le switch ne nécessite aucune configuration pour fonctionner mais apprend dynamiquement la configuration du réseau en observant les paquets reçus et transmis.

Lorsqu'il reçoit une trame adressée à un destinataire qu'il ne connaît pas encore, il la transmet simplement sur toutes ses interfaces. Par la suite, il est probable que la machine venant de recevoir la trame transmette à son tour quelque chose sur le réseau. À ce moment, le switch identifiera l'adresse MAC encore inconnue et l'interface depuis laquelle cette trame a été reçue. Il mettra donc à jour sa table interne pour pouvoir à présent diriger les trames adressées à cette adresse sur la bon interface.

\section{Performances d'un hub et d'un switch}

Objectifs et critères de succès

\subsection{Débits mesurés}

\subsection{Transmission full-duplex et half-duplex}

\subsection{Auto-évaluation}

\end{document}