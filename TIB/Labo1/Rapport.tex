\documentclass[11pt,a4paper]{article}

\usepackage[T1]{fontenc}
\usepackage[utf8]{inputenc}
\usepackage[french]{babel}

\usepackage{fancyhdr}
\usepackage[usenames,dvipsnames]{color}
\usepackage{graphicx}
\usepackage{listings}
\usepackage{titling}
\usepackage{courier}
\usepackage{lipsum}
%\usepackage{showframe}
\usepackage{fullpage}

\topmargin=-10mm
\headsep=5mm
\headheight=10mm

\linespread{1.1}

\setlength\parindent{0pt}

\pagestyle{fancy}
\fancyhf{}
\lhead{\thetitle}
\rhead{\today \\ \theauthor}
\cfoot{\thepage}

\title{TIB-1-B \\ Labo 1: Wireshark}
\author{Bastien Clément \and Christophe Peretti}

\begin{document}

\maketitle

\section*{Problème 1: Wireshark et diagrammes}

\subsection{Objectifs}
Apprendre et effectuer une capture Wireshark simple et documenter les résultats.

Objectifs:

\begin{enumerate}
  \item Effectuer une capture Wireshark
  \item Utiliser un filtre d'affichage
  \item Réaliser un diagramme en flèche
\end{enumerate}

Test

\subsection{Diagramme}

\section*{Problème 2: Image cachée}

\subsection{Objectifs}

Apprendre à utiliser Wireshark pour analyser le trafic réseau et trouver les indices et l'image cachés.

\subsection{Image}

\subsection{Procédure}

\begin{enumerate}
  \item ...
\end{enumerate}


\section{Analyse forensique}

\subsection{Objectifs}

Apprendre à utiliser les fonctions avancées (conversations, flux) de Wireshark. L'objectif est atteint en effectuant l'analyse et en répondant aux questions.

\section{Analyse forensique}

\subsection{Objectifs}

Apprendre à utiliser les fonctions avancées (conversations, flux) de Wireshark. L'objectif est atteint en effectuant l'analyse et en répondant aux questions.

\end{document}