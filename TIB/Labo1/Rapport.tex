\documentclass[11pt,a4paper]{article}

\usepackage[frenchb]{babel}
\usepackage[T1]{fontenc}
\usepackage[utf8]{inputenc}

\title{Labo TIB: Wireshark}
\author{Bastien Clément et Christophe Peretti}

\setlength{\parindent}{0pt}
\setlength{\parskip}{0.5em}

\begin{document}
\maketitle

\section{Wireshark et diagramme en flèches}

\subsection{Objectifs}
Apprendre et effectuer une capture Wireshark simple et documenter les résultats.

Objectifs:
\begin{enumerate}
  \item Effectuer une capture Wireshark
  \item Utiliser un filtre d'affichage
  \item Réaliser un diagramme en flèche
\end{enumerate}

\subsection{Diagramme}

\section{Image cachée}

\subsection{Objectifs}

Apprendre à utiliser Wireshark pour analyser le trafic réseau et trouver les indices et l'image cachés.

\subsection{Image}

\subsection{Procédure}

\begin{enumerate}
  \item ...
\end{enumerate}


\section{Analyse forensique}

\subsection{Objectifs}

Apprendre à utiliser les fonctions avancées (conversations, flux) de Wireshark. L'objectif est atteint en effectuant l'analyse et en répondant aux questions.

\end{document}  