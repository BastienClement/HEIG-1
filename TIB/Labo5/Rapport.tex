\documentclass[11pt,a4paper]{article}

\usepackage[T1]{fontenc}
\usepackage[utf8]{inputenc}
\usepackage[frenchb]{babel}

\usepackage{fancyhdr} % headers
\usepackage[usenames,dvipsnames]{color} % colors
\usepackage{graphicx} % images
\usepackage{listings} % source code
\usepackage{titling} % meta-infos
\usepackage{courier} % courier font
\usepackage{fullpage} % full page layout
\usepackage{titlesec} % title customization
\usepackage{parskip} % paragraphs spacing
\usepackage{amsmath}
\usepackage{tikz}
\usepackage{siunitx}
%\usepackage{showframe} % layout debug

\usepackage{float}
\restylefloat{figure}

\topmargin -10mm
\headsep 5mm
\headheight 10mm

\linespread{1.1}
\renewcommand{\arraystretch}{1.3}

\setlength\parindent{0pt}
\setlength{\unitlength}{1cm}
\setlength{\droptitle}{-1.6cm}

\pagestyle{fancy}
\fancyhf{}
\cfoot{\thepage}

\def \doccourse { TIB1-B }
\def \doctitle {Labo : IP et numéros de port}
\author{Bastien Clément \and Christophe Peretti}

\renewcommand{\thesection}{Objectif \arabic{section} :}
\renewcommand{\thesubsection}{\arabic{section}.\arabic{subsection}}

\rhead{\theauthor \\ \today}
\lhead{\doccourse \\ \doctitle }
\title{{\normalsize \doccourse} \\ \doctitle }

\begin{document}

\maketitle
\vspace{1em}

\section{Analyse des protocoles IP et TCP}

L'objectif de cette première partie est d'apprendre le format des paquet IP et TCP. Il est spécifiquement nécessaire de savoir décrire les champs les plus importants des paquets.

\subsection{Descriptions des champs des paquets IP et TCP}
\subsection{Exemples lors de l'accès à une page web}

\section{Identifier les services d'une machine}

L'objectif de cette partie est d'apprendre à identifier les services qui tournent sur une machine. Il est atteint si nous savons utiliser le logiciel \texttt{nmap} pour énumérer les services actifs sur une machine et que nous connaissons les numéros de ports utilisés par les services les plus courants.

\subsection{Liste des services sur notre machine}
\subsection{Liste des services sur \texttt{10.192.75.14}}
\subsection{Ports courants}

\section{Expérimenter avec les connexions}

L'objectif de cette partie est de s'amuser en explorant l'outil \texttt{netcat}. Par la suite, il nous faudra répondre à quelques questions.

\subsection{Pourquoi faut-il exécuter la commande comme administrateur si l'on veut lancer un serveur sur un port $1$ -- $1023$, mais pas sur les ports supérieurs à $1023$}
\subsection{Est-ce qu'il est possible de lancer deux services qui écoutent le même numéro de port sur la même machine ?}
\footnote{//stackoverflow.com/a/14388707}

\subsection{Et si un utilise TCP et l'autre UDP}

\section{Auto-évaluation}

\end{document}