\documentclass[11pt,a4paper]{article}

\usepackage[T1]{fontenc}
\usepackage[utf8]{inputenc}
\usepackage[frenchb]{babel}

\usepackage{fancyhdr} % headers
\usepackage[usenames,dvipsnames]{color} % colors
\usepackage{graphicx} % images
\usepackage{listings} % source code
\usepackage{titling} % meta-infos
\usepackage{courier} % courier font
\usepackage{fullpage} % full page layout
\usepackage{titlesec} % title customization
\usepackage{parskip} % paragraphs spacing
\usepackage{amsmath}
\usepackage{tikz}
\usepackage{siunitx}
%\usepackage{showframe} % layout debug

\usepackage{float}
\restylefloat{figure}

\topmargin -10mm
\headsep 5mm
\headheight 10mm

\linespread{1.1}
\renewcommand{\arraystretch}{1.3}

\setlength\parindent{0pt}
\setlength{\unitlength}{1cm}
\setlength{\droptitle}{-1.6cm}

\pagestyle{fancy}
\fancyhf{}
\cfoot{\thepage}

\def \doccourse { TIB1-B }
\def \doctitle {Rapport : IP et numéros de port}
\author{Bastien Clément \and Christophe Peretti}

\renewcommand{\thesection}{Objectif \arabic{section} :}
\renewcommand{\thesubsection}{\arabic{section}.\arabic{subsection}}

\rhead{\theauthor \\ \today}
\lhead{\doccourse \\ \doctitle }
\title{{\normalsize \doccourse} \\ \doctitle }

\begin{document}

\maketitle
\vspace{1em}

\section{Analyse des protocoles IP et TCP}

L'objectif de cette première partie est d'apprendre le format des paquet IP et TCP. Il est demandé de savoir décrire les champs les plus importants de ces paquets.

\subsection{Descriptions des champs des paquets IP et TCP}

Le protocole IP est responsable d'acheminer un paquet entre deux machines sur le réseau Internet. L'\textit{en-tête IP}, qui précède le contenu utile du paquet, est composé entre autres de:

\begin{enumerate}
	\item \textbf{Le numéro de version}, permettant de distinguer IPv4 et IPv6.
	\item \textbf{Une somme de contrôle}, permettant de détecter des potentiels erreurs de transmission sur les en-têtes du paquet. En théorie, ce champs n'est pas nécessaire puisque les protocoles de niveau 2 s'assurent de la transmission des frames sans altérations. Cet en-tête n'existe d'ailleurs plus en \textit{IPv6}.
	\item \textbf{L'adresse source et de destination}, permettant d'acheminer le paquet vers le bon destinataire et d'identifier son émetteur.
	\item \textbf{Le protocole de couche supérieure}, permettant à la réception du paquet de le transmettre à la couche supérieure correcte (par exemple TCP ou UDP).
	\item \textbf{Le \textit{time to live}}, un compteur décrémenté à chaque traversée d'un noeud du réseau. Lorsque le compteur atteint 0, le paquet est détruit. Ceci permet d'éviter une boucles de routage infinie qui surchargerait le réseau.
	\item \textbf{Des informations de fragmentation}, permettant la recomposition d'un paquet qui a été découpé au cours de son acheminement sur le réseau.
\end{enumerate}

TCP est quant à lui responsable de fournir une flux de données fiable assurant la réception (en retransmettant les segments perdus) et l'ordre. L'\textit{en-tête TCP} contient entre autres:

\begin{enumerate}
	\item \textbf{Une somme de contrôle}, vérifiant l'intégrité des données.
	\item \textbf{Le port d'origine et de destination}
	\item \textbf{Des numéros de séquence et d'acquittement}, permettant de réordonner correctement des paquets reçus dans le désordre et de détecter les paquets perdus.
	\item \textbf{Des informations de contrôle de flux}, permettant de réguler le débit d'émission en fonction des capacités du récepteur et d'éviter la congestion réseau.
\end{enumerate}

\subsection{Exemples lors de l'accès à une page web}

\begin{tabular}{|l|p{10.3cm}|}
	\hline
	\textbf{En-tête IP} & \textbf{Valeur} \\
	\hline
	Version & 4 \\
	Adresse d'origine & 10.192.74.32 \\
	Adresse de destination & 195.176.255.234 \\
	Protocole & TCP (6) \\
	Time to live & 64 \\
	\hline
	\hline
	\textbf{En-tête TCP} & \textbf{Valeur} \\
	\hline
	Port d'origine & 49633 \\
	Port de destination & http (80) \\
	Sequence number & 1 (\textit{next sequence number}: 328) \\
	Acknowledgement number & 1 \\
	Window size value & 115 (\textit{calculated window size}: 14720, \textit{scaling factor}: 128) \\
	\hline
\end{tabular}

\section{Identifier les services d'une machine}

L'objectif de cette partie est d'apprendre à identifier les services qui tournent sur une machine. Il est atteint si nous savons utiliser le logiciel \texttt{nmap} pour énumérer les services actifs sur une machine et que nous connaissons les numéros de ports utilisés par les services les plus courants.

\subsection{Liste des services sur notre machine}

\begin{verbatim}
PORT    STATE SERVICE VERSION
53/tcp  open  domain  dnsmasq 2.59
631/tcp open  ipp     CUPS 1.5
\end{verbatim}

Nous pouvons observer que notre machine exécute un serveur DNS ainsi qu'un serveur IPP (\textit{Internet Printing Protocol}) utilisé pour le partage d'imprimantes sur le réseau.

\subsection{Liste des services sur \texttt{10.192.75.14}}

\begin{verbatim}
PORT   STATE SERVICE VERSION
21/tcp open  ftp     Brother/HP printer ftpd 1.10
23/tcp open  telnet  Brother/HP printer telnetd
80/tcp open  http    Brother/HP printer webadmin ([...])
\end{verbatim}

Nous sommes manifestement en train de scanner une imprimante ! Les services Telnet et HTTP servent probablement à la configuration et à l'administration de l'appareil. Nous pouvons en revanche nous interroger sur les raisons pour une imprimante de proposer un service FTP ?

\subsection{Ports courants}

\begin{tabular}{|l|l|p{12cm}|}
	\hline
	\textbf{Port} & \textbf{Service} & \textbf{Description} \\
	\hline
	80    & HTTP & Utilisé par les serveurs Web pour les requêtes HTTP \\
	20, 21 & FTP & Transmission de fichiers (port de données / contrôle) \\
	22    & SSH & Terminal distant sécurisé (mais aussi SFTP / SCP / TCP tunneling) \\
	25    & SMTP & Envoi et relai d'e-mail \\
	53    & DNS & Résolution de nom en adresse IP (et autres services basés sur DNS) \\
	443   & HTTPS & HTTP sécurisé (cryptage SSL / TLS) \\
	\hline
\end{tabular}

\section{Expérimenter avec les connexions}

L'objectif de cette partie est de s'amuser en explorant l'outil \texttt{netcat}. Par la suite, il nous faudra répondre à quelques questions.

\subsection{Pourquoi faut-il les droits d'administrateur si l'on veut lancer un serveur sur un port $1$ -- $1023$, mais pas sur les ports supérieurs à $1023$ ?}

Puisque les logiciels utilisant les \textit{well-known ports} (1 -- 1023) sont généralement des processus en \textit{background} exécutés par le système, ils disposent déjà des droits administrateur requis pour ouvrir un tel port. Ceci implique également qu'il est impossible pour un utilisateur classique de mettre en place un serveur sur une machine sans l'accord de l'administrateur.

\subsection{Est-ce qu'il est possible de lancer deux services qui écoutent le même numéro de port sur la même machine ?}

En principe non. Il serait en effet impossible au système de savoir auquel des deux processus une connexion sur le port en question devrait être transmise.  En pratique, les systèmes d'exploitations offrent des fonctionnalités permettant à plusieurs processus de partager le même port.

%Une fois la connexion établie, l'adresse et le port distants permettent alors de différencier les connexions sans ambiguïté. 

% stackoverflow.com/a/14388707

\subsection{Et si un utilise TCP et l'autre UDP ?}

Ce n'est alors plus un problème. Le protocole utilisé fait partie des critères permettant de distinguer les connexions. Le système d'exploitation n'a aucune difficulté à identifier les processus utilisant l'un ou l'autre des protocoles.

\section{Auto-évaluation}

Nous considérons avoir atteint les objectifs de ce laboratoire.

\end{document}