\documentclass[11pt,a4paper]{article}

\usepackage[T1]{fontenc}
\usepackage[utf8]{inputenc}
\usepackage[frenchb]{babel}

\usepackage{fancyhdr} % headers
\usepackage[usenames,dvipsnames]{color} % colors
\usepackage{graphicx} % images
\usepackage{listings} % source code
\usepackage{titling} % meta-infos
\usepackage{courier} % courier font
\usepackage{fullpage} % full page layout
\usepackage{titlesec} % title customization
\usepackage{parskip} % paragraphs spacing
\usepackage{amsmath}
\usepackage{tikz}
\usepackage{siunitx}
%\usepackage{showframe} % layout debug

\usepackage{float}
\restylefloat{figure}

\topmargin -10mm
\headsep 5mm
\headheight 10mm

\linespread{1.1}
\renewcommand{\arraystretch}{1.3}

\setlength\parindent{0pt}
\setlength{\unitlength}{1cm}
\setlength{\droptitle}{-1.6cm}

\pagestyle{fancy}
\fancyhf{}
\cfoot{\thepage}

\def \doccourse { TIB1-B }
\def \doctitle {Projet: configuration d'un réseau complet}
\author{Bastien Clément \and Christophe Peretti}

%\renewcommand{\thesection}{Partie \arabic{section} :}
\renewcommand{\thesubsection}{\arabic{section}.\arabic{subsection}}

\rhead{\theauthor \\ \today}
\lhead{\doccourse \\ \doctitle }
\title{{\normalsize \doccourse} \\ \doctitle }

\begin{document}

\maketitle
\vspace{1em}

\section{Diagramme du réseau établi}

\section{Plan d'adressages}

\subsection{Plan d'adressage des sous-réseaux IPv4}

Selon la donnée, notre plan d'adressage doit:

\begin{enumerate}
	\item Utiliser une plage d'adresse privée
	\item Permettre au minimum 10 sous-réseaux
	\item Permettre au minimum 200 machines par sous-réseaux
\end{enumerate}

Nous avons donc choisi d'utiliser la plage privée \texttt{10.0.0.0/8} qui est la plus large des plages d'adresse privées possible. Sa capacité est à priori largement supérieure aux besoins, mais elle a l'avantage d'être simple à retenir et de permettre de sous-diviser facilement les réseaux si le besoin devait survenir dans le futur. Puisqu'il s'agit d'adresse privées, il n'y a pas de gaspillage.

Chaque sous-réseau utilise le masque \texttt{255.255.0.0}. Permettant ainsi 256 sous-réseaux et plus de 65000 machines par sous-réseaux.

L'attribution des préfixes des sous-réseau est alors très simple:

\begin{tabular}{|l|l|}
	\hline
	\textbf{Sous-réseau} & \textbf{Préfixe} \\
	\hline
	Sous-réseau 1 & 10.1.0.0 \\
	Sous-réseau 2 & 10.2.0.0 \\
	Sous-réseau 3 & 10.3.0.0 \\
	... & ... \\
	Sous-réseau 9 & 10.9.0.0 \\
	Sous-réseau 10 & 10.10.0.0 \\
	\hline
\end{tabular}
\vspace{1em}

Nous conservons le sous-réseau \texttt{10.0.0.0/16} pour l'interconnexion des routeurs eux-mêmes.

\subsection{Interconnexion des sous-réseaux}

\subsection{Adressage IPv6}

\section{Fichiers de configuration des routeurs}

\section{Auto-évaluation}

Nous considérons avoir atteint les objectifs de ce laboratoire.

\end{document}