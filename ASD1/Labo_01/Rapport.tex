\documentclass[11pt,a4paper]{article}

\usepackage[T1]{fontenc}
\usepackage[utf8]{inputenc}
\usepackage[frenchb]{babel}

\usepackage{fancyhdr} % headers
\usepackage[usenames,dvipsnames]{color} % colors
\usepackage{graphicx} % images
\usepackage{listings} % source code
\usepackage{titling} % meta-infos
\usepackage{courier} % courier font
\usepackage{fullpage} % full page layout
\usepackage{titlesec} % title customization
\usepackage{parskip} % paragraphs spacing
\usepackage{listings}
%\usepackage{showframe} % layout debug

\topmargin=-10mm
\headsep=5mm
\headheight=10mm

\linespread{1.0}

\setlength\parindent{0pt}
\setlength{\unitlength}{1cm}
\setlength{\droptitle}{-1.6cm}

\lstset{
	tabsize=4,
	frame=single,
	language=Pascal
}

\pagestyle{fancy}
\fancyhf{}
\cfoot{\thepage}

\def \doctitle { ASD1-D \\ Labo 1: Évaluation de polynômes }
\author{Bastien Clément \and Christophe Peretti}

\rhead{\theauthor \\ \today}
\lhead{\doctitle }
\title{\doctitle }

\begin{document}

\maketitle

\section{Partie théorique}

\subsection{Algorithme 1: $ P(X_{0}) = \sum\limits_{i=0}^{n} a_{i} \cdot X_{0}^{i} $}

Ce premier algorithme correspond à la transcription directe et naïve de la somme des termes correspondants à chaque degré pour calculer la valeur du polynôme.

\begin{lstlisting}
var value = 0
for i = 0 to n do
	term = a[i]
	for j = 0 until i do
		term = term * x
	end
	value = value + term
end
\end{lstlisting}

\subsection{Algorithme 2: Optimisation $ X_{0}^{i+1} = X_{0}^{i+1} \cdot X_{0} $}

Une optimisation évidente du premier algorithme consiste à ne pas recommencer l'exponentiation de la variable x de 0 à chaque terme. Mémoriser la puissance précédente et la multiplier une fois de plus est suffisant.

\begin{lstlisting}
var value = 0
var term = 1
for i = 0 to n do
	value = value + ( a[i] * term )
	term = term * x
end
\end{lstlisting}

\subsection{Algorithme 3: $ P(X_{0}) = a_{0} + X_{0} \cdot (a_{1} + X_{0} \cdot (a_{2} + \cdots + X_{0} \cdot (a_{n-1} + X_{0} \cdot a_{n}))) $}

\begin{lstlisting}
var value = a[n]
for i = n-1 to 0 do
	value = (value * x) + a[i]
end
\end{lstlisting}

\subsection{Comparaisons}

\end{document}